\section{Описание модуля}
Модуль \verb"geometry" представляет из себя подключаемый пакет на языке {\sf C++}. Состоит из двух файлов \verb"geometry.hpp" и \verb"geometry.cpp". В модуле реализован набор классов, отвечающих за базовые пространственные фигуры (примитивы), преобразование фигур (сдвиг, поворот, трансформация) и алгебры битовых операций для комбинирования нескольких фигур (объединение, пересечение, исключение). Модуль использует в качестве подключаемого пакета библиотеку {\sf aivlib}. Проекты с подключенным модулем можно также компилировать с использованием {\sf SWIG 2}.

В основе пакета лежит технология <<Конструктивная сплошная геометрия>> (Constructive Solid Geometry, CSG)~\cite{csg}. 
НАследник базового типа \verb"BaseFigure".

\section{Примитвы}

\subsection{Сфера}
\subsection{Цилинлр}
\subsection{Прямоугольный параллелограмм}

\section{Преобразование фигур}

\subsection{Сдвиг}
\subsection{Поворот}
\subsection{Трансформация}

\section{Алгебра битовых операций}

\subsection{Объединение}
\subsection{Пересечение}
\subsection{Исключение}

\newpage
\begin{thebibliography}{9}
    \bibitem{csg} \textit{Foley, James D.} (1996), "12.7 Constructive Solid Geometry", Computer Graphics: Principles and Practice, Addison-Wesley Professional, pp. 557–558, ISBN 9780201848403.
\end{thebibliography}
